\chapter[Software para flujos de trabajo]{Software para la administración y ejecución de flujos de trabajo}

En el capítulo anterior se extendió la clasificación de algoritmos de planificación elaborada por Yu et al. \cite{yu2008workflow}. Sin duda, estos algoritmos han sido probados en simulaciones e implementados en sistemas que administran la ejecución de los flujos de trabajo. De acuerdo a Yu el al. \cite{yu2008workflow}, un sistema de administración de flujos de trabajo\footnote{En la literatura académica, estos sistemas son conocidos como \emph{workflow management software systems}} se encarga de definir, coordinar y ejecutar los flujos de trabajo en los recursos de cómputo.

Para este trabajo, clasificaremos los sistemas de administración de flujos de trabajo de acuerdo a su enfoque de cómputo, enumerando las siguientes características: año de aparición, proyecto, utilización, autores y los algoritmos de planificación utilizados en estos sistemas.

\section{Software orientado a clusters}

%\subsection{Open Grid Scheduler}
%NO entra!!!, no trabaja con flujos de trabajo o si?

\subsection{HTCondor}
%URL: \url{http://research.cs.wisc.edu/htcondor/}

HTCondor \cite{condor-practice} es un sistema para administrar flujos de trabajo que necesitan ser ejecutados en entornos de cómputo distribuido. Desde 1983 \cite{htcondor2014webpage}, HTCondor es desarrollado y mantenido por la Universidad de Wisconsin en Madison. Éste tiene un módulo llamado DAGMan que se encarga de planificar tareas de un flujo de trabajo, expresadas como grafos dirigidos acíclicos. El algoritmo de planificación utilizado por DAGMan es el algoritmo Miope. 

HTCondor presenta al usuario un único recurso de cómputo, formado de varias computadoras interconectadas entre sí. Esta es la razón por la que este sistema de administración de flujos de trabajo cae dentro de la clasificación de software orientado a clusters.

\section{Software orientado a grids}

\subsection{SwinDew-G}
%URL: \url{http://www.swinflow.org/swindew/grid/}

%URL: \url{http://link.springer.com/chapter/10.1007%2F978-1-4614-1933-4_5}

SwinDew-G \cite{yang2007peer} es un sistema de administración de flujos de trabajo cuya característica especial es que trabaja con redes de computadoras P2P. En la primera versión de SwinDew, el proceso de planificación es estático \cite{yang2007peer}, i.e., en la fase de preparación de la ejecución se crea el plan de ejecución del flujo y se aplica sin ninguna modificación a lo largo del tiempo. Varios esfuerzos se han hecho para mejorar el mecanismo de planificación, como el algoritmo CTC \cite{liu2010compromised} diseñado especialmente para flujos de trabajo que son intensivos en instancias pero que contienen pocas tareas simples, como los flujos de trabajo utilizados en los bancos y empresas \cite{liu2011novel}.

%También, existen versiones específicas de SwinDew para nubes. En este caso, citamos la versión para grids llamada SwinDew-G. La primera versión de SwinDew-G aparece en publicaciones aceptadas en el 2006.
Finalmente, SwinDew y sus variantes fueron desarrollados en la Universidad de Swinburne, en Australia.

\subsection{Pegasus}
%URL: \url{https://pegasus.isi.edu/}

Pegasus \cite{deelman2005pegasus} fue desarrollado en la Universidad del Sur de California. Al igual que SwinDew-G, se puede utilizar Pegasus para trabajar con grids y con nubes. Pegasus implementa MaxMin, MinMin y Sufragio como algoritmos de planificación. Las principales aplicaciones de Pegasus son los flujos de trabajo científicos. Los primeros trabajos publicados que reportan el uso de Pegasus datan del 2003 \cite{pegasus2014publications}. Dentro de la arquitectura de Pegasus, el módulo encargado de planificar el flujo de trabajo es el Mapper.

\section{Software orientado a nubes}

\subsection{SwinDew-C}

SwinDew-C \cite{liu2010swindew} es un sistema de administración de flujos de trabajo basado en SwinDew-G, cuya principal característica es que agrega nubes externas como recursos disponibles para la ejecución de flujos de trabajo. La arquitectura de SwinDew-C es muy similar a la arquitectura de SwinDew-G, con la diferencia de que agrega nuevos componentes para manejar restricciones de calidad de servicio y administración de fallos. En el caso de una falla en la ejecución de las tareas, SwinDew-C replanifica la ejecución de la tarea fallida con un algoritmo basado en optimización por colonia de hormigas.
%checar referencia en el paper de SwinDew-C para colonia de hormigas

\subsection{Aneka}
%URL: \url{http://www.manjrasoft.com/products.html}

Aneka \cite{chu2007aneka} es una implementación de una plataforma como servicio que, además de poder especificar y ejecutar flujos de trabajo, también puede planificar y ejecutar aplicaciones distribuidas que hayan sido construidas con distintos modelos de programación distribuida: basado en tareas, basado en threads y basado en MapReduce. El modelo de programación que se utiliza en Aneka para trabajar con flujos de trabajo es el modelo basado en tareas.
%En el primer paper, de grids, se usan estos modelos: task farming y dataflow model.

El diseño de Aneka es basado en servicios. Cada servicio se encarga de una funcionalidad específica para la administración del grid o nube. Así, existe un módulo específico para la planificación de la ejecución de las aplicaciones que varía de acuerdo al modelo de programación elegido para desarrollar la aplicación.

Técnicamente, es posible construir una nube con Aneka \cite{vecchiola2009aneka} utilizando computadoras de escritorio convencionales. De esta forma, Aneka provee servicios para administrar el esquema de precios de la nube construida con este software. Del mismo modo, Aneka utiliza un mecanismo de negociación mediante agentes que buscan \emph{cerrar} los mejores tratos con los proveedores de los servicios de la nube, que mejor satisfagan los requisitos de calidad en el servicio.

Es importante mencionar que la plataforma Aneka fue desarrollada inicialmente por el laboratorio GRIDS de la Universidad de Swinburne, en Australia. Actualmente, Aneka es desarrollado por la empresa Manjrasoft \cite{aneka2014webpage}.

\subsection{Askalon}
%URL: \url{http://www.dps.uibk.ac.at/projects/askalon/}

Askalon \cite{fahringer2005askalon} es un entorno que facilita el desarrollo y la ejecución de flujos de trabajo. Askalon utiliza un lenguaje llamado AGWL para especificar flujos de trabajo. Al igual que Aneka, el diseño de Askalon está basado en servicios, tales como: el motor de promulgación, el administrador de recursos, la base de datos de checkpoints y el planificador.

En el servicio planificador, Askalon utiliza varios algoritmos de planificación: HEFT, Miope y un algoritmo genético. Esto es porque antes de ejecutar el flujo de trabajo, se hace una predicción del tiempo estimado de ejecución de cada tarea. Entonces, hay un módulo encargado de actualizar la información de las predicciones y, en base al resultado de los tres algoritmos, se elige la planificación que mejor cumpla con los requisitos de calidad en el servicio dictados por el usuario.

Actualmente, el proyecto Askalon es desarrollado por la Universidad de Innsbruck, en Austria \cite{askalon2014webpage}.