\begin{center}
Resumen
\end{center}
\noindent Un flujo de trabajo es un conjunto de pasos que modelan la ejecución de un proceso. Éstos son utilizados para modelar aplicaciones como la anotación de proteínas. Normalmente, estas aplicaciones requieren vastos recursos computacionales, por lo que se han empleado enfoques de cómputo distribuido (clusters, grids y nubes) para ejecutar estos flujos de trabajo. Dos aspectos importantes para la ejecución de los flujos de trabajo son la planificación de las tareas y la asignación de recursos distribuidos. Así, el objetivo de la planificación de un flujo de trabajo es 1) maxaimizar el uso de los recursos distribuidos, 2) minimizar el tiempo total de ejecución del flujo de trabajo o 3) reducir el costo de ejecución del flujo de trabajo. En esta tesis, se elaboró una taxonomía de algoritmos de planificación de flujos de trabajo, con base en la función objetivo que optimiza cada algoritmo, e.g., optimización del costo. También, se elaboró un simulador de ejecución de flujos de trabajo para estudiar y comparar los algoritmos de planificación de flujos de trabajo. Por último, se realizó un análisis empírico comparativo de tres algoritmos de planificación para determinar en qué casos son adecuados cada algoritmo.
\\\\
\noindent \emph{Palabras clave}: Planificación de flujos de trabajo, Cómputo distribudo, Algoritmos de planificación

\begin{center}
Abstract
\end{center}
\noindent A workflow is a set of steps that model the execution of a process. For instance, an application commonly modeled as a workflow is the annotation of proteins. Typically, workflow applications require vast computational resources; thus, distributed computing approaches (such as clusters, grids and clouds) have been used to execute them. Important aspects of workflow execution are task scheduling and resource allocation. The objective of workflow scheduling is to 1) maximize the use of distributed resources, 2) minimize workflow makespan (amount of time required to complete a workflow), or 3) reduce workflow execution cost. In this thesis, a taxonomy of workflow scheduling algorithms was made, The taxonomy classifies workflow scheduling algorithms according to the type of the objective function being optimized, e.g., cost optimization. Also, a workflow execution simulator was developed in order to study and compare workflow scheduling algorithms. Finally, An empirical analysis of three commonly used workflow scheduling algorithms was conducted to determine under what circumstances each scheduling algorithm is most efficient.
\\\\
\noindent \emph{Keywords}: Workflow scheduling, Distributed Computing, Scheduling algorithms

\clearpage
