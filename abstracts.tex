\begin{center}
Resumen
\end{center}
\noindent Los flujos de trabajo son un conjunto de tareas que modelan la ejecución de un proceso. Éstos son utilizados para modelar aplicaciones como la anotación de proteínas o el ajuste de cuentas bancarias. Normalmente, estas aplicaciones requieren vastos recursos computacionales, por lo se han empleado enfoques de cómputo distribuido (clusters, grids y nubes) para ejecutar estos flujos de trabajo. Una parte importante para la ejecución de los flujos de trabajo es la planificación de las tareas en los recursos distribuidos. Con ello, se pueden hacer planificaciones que maximicen el uso del sistema distribuido o que minimicen el tiempo total de ejecución. En esta tesis, se elaboró una clasificación de algoritmos de planificación de flujos de trabajo, basada en la forma en que los algoritmos optimizan la planificación. También, se elaboró un simulador de ejecución de flujos de trabajo para estudiar y comparar los algoritmos de planificación de flujos de trabajo. Por último, se realizó un estudio comparativo de tres algoritmos de planificación para determinar en qué casos son adecuados cada algoritmo.
\\\\
\noindent \emph{Palabras clave}: Planificación de flujos de trabajo, Cómputo distribudo, Algoritmos de planificación

\begin{center}
Abstract
\end{center}
\noindent Workflows are a set of tasks that model the execution of a process. These are used to model applications like protein annotation or making the balance of banking accounts. Typically, these applications require vast computational resources; thus, distributed computing approaches (i.e. clusters, grids and clouds) have been used to execute these workflows. An important part of the workflow execution is the planning of the tasks on distributed resources. This execution plan, or schedule, could maximize the use of distributed resources or it could minimize makespan. In this thesis, a classification of workflow scheduling algorithms was developed, based on the way that algorithms optimizes the schedule. Also, a workflow execution simulator was developed in order to study and compare the workflow scheduling algorithms. Finally, a study of three scheduling algorithms is performed to determine in which cases are suitable each algorithm.
\\\\
\noindent \emph{Keywords}: Workflow scheduling, Distributed Computing, Scheduling algorithms

\clearpage
