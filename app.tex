\chapter{Pseudocódigos de algoritmos de planificación de flujos de trabajo}

\section{Algoritmos de mejor esfuerzo}
\subsection{Miope}
El algoritmo miope \cite{yu2008workflow} es el más simple de todos los algoritmos. Lo único que hace es buscar un recurso disponible que pueda ejecutar la tarea y asignarle dicha tarea. No toma en cuenta otra característica a optimizar. Este algoritmo fue propuesto por Ramamritham et al. \cite{ramamritham1990efficient}. 
%El algoritmo peresentado en estre trabajo se encuentra en la sección \ref{alg:myopic} del Apéndice.

\label{alg:myopic}
\begin{algorithmic}[1]
\Require Un grafo de flujo de trabajo $w=(\mathcal{V}, \mathcal{E})$
\Ensure{Una calendarización $f$}
\While{$\exists v \in \mathcal{V}$ no completada}
	\State $t \gets$ Obtener una tarea lista, no calendarizada, con padres calendarizados;
	\State $r \gets$ Obtener un recurso que pueda resolver la tarea en el menor tiempo;
	\State Calendarizar $t$ en $r$, i.e., $f(r) = t$;
\EndWhile
\end{algorithmic}

\subsection{Definiciones para los algoritmos Max-Min y Min-min}
Los algoritmos min-min y max-min utilizan las siguientes  estimaciones: %definidas en la tabla \ref{tbl:min-min-estimators}. %todo: checar por qué numera RARO
\begin{itemize}
\item{$EET(t,r)$ -- \textbf{Tiempo estimado de ejecución:} Tiempo que el recurso (servicio) $r$ tomará en ejecutar la tarea $t$, desde que la tarea es ejecutada en el recurso}
\item{$EAT(t,r)$ -- \textbf{Tiempo estimado de disponibilidad:} Tiempo en el que el recurso $r$ estará disponible para ejecutar la tarea $t$}
\item{$FAT(t,r)$ -- \textbf{Tiempo de archivo disponible:} Tiempo más pronto en que todos los archivos requeridos por la tarea $t$ están disponibles en el recurso $r$}
\item{$ECT(t,r)$ -- \textbf{Tiempo estimado de terminación:} Tiempo estimado en cual la tarea $t$ terminará su ejecución en el recurso $r$: 
              \[ ECT(t,r) = EET(t,r) + max(EAT(t,r), FAT(t,r)) \]}
\item{$MCT(t)$ -- \textbf{Tiempo mínimo estimado de terminación: } $ECT$ mínimo para la tarea $t$ sobre todos los recursos disponibles, es decir: 
            \[ MCT(t) = \min_{r \in \mathcal{S}} ECT(t,r) \]}
\end{itemize}

\subsection{Min-Min}
El algoritmo min-min está basado en la heurística de terminar las tareas más cortas en el menor tiempo posible. Para ello, hace una estimación del tiempo de ejecución tomando en cuenta el tiempo de preparación de las tareas en los servicios --o recursos-- para tener el tiempo y los archivos necesarios disponibles para el recurso en cuestión. El algoritmo fue presentado por Maheswaran et al. \cite{maheswaran1999dynamic}. 
%El pseudocódigo está descrito en el listado \ref{alg:min-min}.

\label{alg:min-min}
\begin{algorithmic}[1]
\Require{Un grafo de flujo de trabajo $w=(\mathcal{V}, \mathcal{E})$}
\Ensure{Una calendarización $f$}
\While{$\exists v \in \mathcal{V}$ \emph{no completada}}
	\State $t \gets$ Obtener conjunto de tareas listas, no calendarizadas, con padres  calendarizados;
	\State \Call{Sched}{$tasks$};
\EndWhile
\Procedure{Sched}{$availTasks$}
	\While{$\exists t \in availTasks$ no calendarizadas}
		\For{$t \in availTasks$}
			\State $res \gets$ Obtener recursos disponibles para $t$;
			\For{$r \in res$}
				\State Calcular $ECT(t,r)$;
			\EndFor
			\State $R_T \gets \arg\min_{r \in res}ECT(t,r)$;
		\EndFor
		\State $T \gets \arg\min_{t \in availTasks}ECT(t,R_T)$;
		\State Calendarizar $T$ en $R_T$;
		\State Remover $T$ de \emph{availTasks};
		\State Actualizar $EAT(R_T)$;
	\EndWhile
\EndProcedure
\end{algorithmic}

\subsection{Max-Min}
El algoritmo max-min --también propuesto Maheswaran et al. \cite{maheswaran1999dynamic}-- es muy similar al algoritmo min-min. La diferencia radica en que éste calendariza tareas cuyo tiempo mínimo de ejecución es el mayor, de tal modo que se ejecutan las tares más \emph{largas}.

El único cambio que se necesita hacer al algoritmo \ref{alg:min-min} es cambiar la siguiente línea (14):
\[T \leftarrow \arg\min_{t \in \emph{availTasks}}ECT(t,r);\]
por
\[T \leftarrow \arg\max_{t \in \emph{availTasks}}ECT(t,r);\]
Así, el algoritmo modificado puede verse en el pseudocódigo \ref{alg:max-min}.

\label{alg:max-min}
\begin{algorithmic}[1]
\Require{Un grafo de flujo de trabajo $w=(\mathcal{V}, \mathcal{E})$}
\Ensure{Una calendarización $f$}
\While{$\exists v \in \mathcal{V}$ \emph{no completada}}
	\State $t \gets$ Obtener conjunto de tareas listas, no calendarizadas, con padres  calendarizados;
	\State \Call{Sched}{$tasks$};
\EndWhile
\Procedure{Sched}{$availTasks$}
	\While{$\exists t \in availTasks$ no calendarizadas}
		\For{$t \in availTasks$}
			\State $res \gets$ Obtener recursos disponibles para $t$;
			\For{$r \in res$}
				\State Calcular $ECT(t,r)$;
			\EndFor
			\State $R_T \gets \arg\min_{r \in res}ECT(t,r)$;
		\EndFor
		\State $T \gets \arg\max_{t \in availTasks}ECT(t,R_T)$;
		\State Calendarizar $T$ en $R_T$;
		\State Remover $T$ de \emph{availTasks};
		\State Actualizar $EAT(R_T)$;
	\EndWhile
\EndProcedure
\end{algorithmic}

\subsection{Sufragio}
El algoritmo Sufragio\footnote{También conocido como \emph{Sufferage}} \cite{maheswaran1999dynamic} es una variación del algoritmo Min-min, el cual considera el valor del sufragio para hacer la calendarización. Dicho valor es la diferencia entre el menor tiempo de ejecución para una tarea $t$ sobre un conjunto de recursos disponibles y el segundo menor. Se calendariza a la tarea que tenga el valor del sufragio más alto, por el hecho de que las tareas que son muy sensibles a los cambios de los recursos deben ser calendarizadas primero. 
%El algoritmo se encuentra en la sección \ref{alg:sufferage} del Apéndice.

\label{alg:sufferage}
\begin{algorithmic}[1]
\Require{Un grafo de flujo de trabajo $w=(\mathcal{V}, \mathcal{E})$}
\Ensure{Una calendarización $f$}
\While{$\exists v \in \mathcal{V}$ \emph{no completada}}
	\State $t \gets$ Obtener conjunto de tareas listas, no calendarizadas, con padres  calendarizados;
	\State \Call{Sched}{$tasks$};
\EndWhile
\Procedure{Sched}{$availTasks$}
	\While{$\exists t \in availTasks$ no calendarizadas}
		\For{$t \in availTasks$}
			\State $res \gets$ Obtener recursos disponibles para $t$;
			\For{$r \in res$}
				\State Calcular $ECT(t,r)$;
			\EndFor
			\State $R^1_t \gets \arg\min_{r \in res}ECT(t,r)$;
			\State $R^2_t \gets \arg\min_{r \in res, r \ne{R^1_t}}ECT(t,r)$;
			\State $suf_t \gets ECT(t,R^2_t) - ECT(t,R^1_t)$;
		\EndFor
		\State $T \gets \arg\max_{t \in availTasks}suf_t$;
		\State Calendarizar $T$ en $R^1_T$;
		\State Remover $T$ de \emph{availTasks};
		\State Actualizar $EAT(R_T)$;
	\EndWhile
\EndProcedure
\end{algorithmic}

\subsection{HEFT}
El algoritmo HEFT\footnote{Heterogeneous Earliest-Finish Time} fue propuesto por Topcuoglu et al. \cite{topcuoglu2002performance}

\label{alg:heft}
\begin{algorithmic}[1]
\Require{Un grafo de flujo de trabajo $w=(\mathcal{V}, \mathcal{E})$}
\Ensure{Una calendarización $f$}
\State Calcular \emph{Tiempo de Ejecución Promedio} (1) para cada tarea $v \in \mathcal{V}$;
\State Calcular \emph{Tiempo de Transferencia de Datos Promedio} (2) entre tareas y sus sucesores;
\State Calcular \emph{Rango} para cada tarea, de acuerdo a (3) (4);
\State Ordenar las tareas por \emph{Rango}, en orden decreciente en una lista $Q$;
\While{$Q \ne \varnothing$}
	\State $t \gets$ Remover la primera tarea de $Q$;
	\State $r \gets$ Encontrar un recurso que pueda ejecutar $r$ en el menor tiempo;
	\State Calendarizar $t$ en $r$;
\EndWhile
\end{algorithmic}

\subsection{Híbrido}
\label{alg:hybrid}
\begin{algorithmic}[1]
\Require{Un grafo de flujo de trabajo $w=(\mathcal{V}, \mathcal{E})$}
\Ensure{Una calendarización $f$}
\State Calcular \emph{Peso} de cada tarea y arista, de acuerdo a (1)(2);
\State Calcular \emph{Rango} para cada tarea, de acuerdo a (3) (4);
\State Ordenar las tareas por \emph{Rango}, en orden decreciente en una lista $Q$;
\State $i \gets 0$
\State Crear un grupo $G_i$;
\While{$Q \ne \varnothing$}
	\State $t \gets$ Remover la primera tarea de $Q$;
	\If{$t$ tiene una dependencia con una tarea en $G_i$}
		\State $i \gets i + 1$;
		\State Crear un grupo $G_i$;
	\EndIf
	\State Agregar $t$ a $G_i$;
\EndWhile
\State $j \gets 0$;
\While{$j \le i$}
	\State Calendarizar tareas en $G_i$ usando un algoritmo \emph{batch};
	\State $j \gets j + 1$;
\EndWhile
\end{algorithmic}

\subsection{TANH}
\label{alg:tanh}
\begin{algorithmic}[1]
\Require{Un grafo de flujo de trabajo $w=(\mathcal{V}, \mathcal{E})$}
\Ensure{Una calendarización $f$}
\State Calcular \emph{Parámetros} para cada nodo tarea;
\State Agrupar tareas del flujo de trabajo;
\If{Número de clusters $\ge$ Número de recursos disponibles}
	\State Reducir el Número de clusters al Número de recursos disponibles;
\Else
	\State Ejecutar duplicación de tareas;
\EndIf
\end{algorithmic}

%\section{Algoritmos de Calidad en el Servicio}
%\subsection{BackTracking}

%\subsection{Deadline Distribution}
%\subsection{CTC}
%\subsection{Deadline-MDP}
%\subsection{LOSS/GAIN}

\section{Metaheurísticos}
%\subsection{Algoritmo Genético}

\subsection{GRASP}
\label{alg:grasp}
\begin{algorithmic}[1]
\Require{Un grafo de flujo de trabajo $w=(\mathcal{V}, \mathcal{E})$}
\Ensure{Una calendarización $f$}
\While{Criterio de terminación no satisfactorio}
	\State $sched \gets$ \Call{createSchedule}{$w$};
	\If{$sched$ es mejor que $bestSched$}
		\State $bestSched \gets sched$;
	\EndIf
\EndWhile
\Procedure{createSchedule}{$workflow$}
	\State $solution \gets$ \Call{constructSolution}{$workflow$};
	\State $nSolution \gets$ \Call{localSearch}{$solution$};
	\If{$nSolution$ es mejor que $solution$}
		\State \textbf{return} $nSolution$;
	\EndIf
	\State \textbf{return} $solution$;
\EndProcedure
\Procedure{constructSolution}{$workflow$}
	\While{calendarización no completada}
		\State $T \gets$ Obtener tareas listas sin asignar;
		\State Crear RCL para cada $t \in T$;
		\State $subSolution \gets$ Seleccionar un recurso \emph{aleatoriamente} para cada $t \in T$ de su RCL;
		\State $solution \gets solution \cup subSolution$;
		\State Actualizar información para futuros RCL;
	\EndWhile
	\State \textbf{return} $solution$;
\EndProcedure
\Procedure{localSearch}{$solution$}
	\State $nSolution \gets$ Encontar una solución local óptima;
	\State \textbf{return} $nSolution$;
\EndProcedure
\end{algorithmic}

%\subsection{Recocido simulado}
