\chapter[Pseudocódigos]{Pseudocódigos de algoritmos de planificación de flujos de trabajo}

\section{Algoritmos de mejor esfuerzo}

\subsection{Miope}
\label{alg:myopic}
El algoritmo Miope \cite{yu2008workflow} es el más simple de todos los algoritmos. Lo único que hace es buscar un recurso disponible que pueda ejecutar la tarea y asignarle dicha tarea. No toma en cuenta otra característica a optimizar. Este algoritmo fue propuesto por Ramamritham et al. \cite{ramamritham1990efficient}. 
%El algoritmo peresentado en estre trabajo se encuentra en la sección \ref{alg:myopic} del Apéndice.


\begin{algorithmic}[1]
\Require Un grafo de flujo de trabajo $w=(\mathcal{V}, \mathcal{E})$
\Ensure{Una planificación $f$}
\While{$\exists v \in \mathcal{V}$ no completada}
	\State $t \gets$ Obtener una tarea lista, no planificada, con padres planificados;
	\State $r \gets$ Obtener un recurso que pueda empezar la tarea en el menor tiempo;
	\State Planificar $t$ en $r$, i.e., $f(r) = t$;
\EndWhile
\end{algorithmic}

\subsection{Definiciones para los algoritmos MaxMin y MinMin}
\label{alg:def_maxmin}
Los algoritmos MinMin y MaxMin utilizan las siguientes estimaciones: %definidas en la tabla \ref{tbl:min-min-estimators}. %todo: checar por qué numera RARO
\begin{itemize}
\item{$EET(t,r)$ -- \textbf{Tiempo estimado de ejecución:} Tiempo que el recurso (servicio) $r$ tomará en ejecutar la tarea $t$, desde que la tarea es ejecutada en el recurso}
\item{$EAT(t,r)$ -- \textbf{Tiempo estimado de disponibilidad:} Tiempo en el que el recurso $r$ estará disponible para ejecutar la tarea $t$}
\item{$FAT(t,r)$ -- \textbf{Tiempo de archivo disponible:} Tiempo más pronto en que todos los archivos requeridos por la tarea $t$ están disponibles en el recurso $r$}
\item{$ECT(t,r)$ -- \textbf{Tiempo estimado de terminación:} Tiempo estimado en cual la tarea $t$ terminará su ejecución en el recurso $r$: 
              \[ ECT(t,r) = EET(t,r) + max(EAT(t,r), FAT(t,r)) \]}
\item{$MCT(t)$ -- \textbf{Tiempo mínimo estimado de terminación: } $ECT$ mínimo para la tarea $t$ sobre todos los recursos disponibles, es decir: 
            \[ MCT(t) = \min_{r \in \mathcal{S}} ECT(t,r) \]}
\end{itemize}

\subsection{MinMin}
\label{alg:min-min}
El algoritmo MinMin está basado en la heurística de terminar las tareas más cortas en el menor tiempo posible. Para ello, hace una estimación del tiempo de ejecución tomando en cuenta el tiempo de preparación de las tareas en los servicios --o recursos-- para tener el tiempo y los archivos necesarios disponibles para el recurso en cuestión. El algoritmo fue propuesto por Maheswaran et al. \cite{maheswaran1999dynamic}. 
%El pseudocódigo está descrito en el listado \ref{alg:min-min}.

\begin{algorithmic}[1]
\Require{Un grafo de flujo de trabajo $w=(\mathcal{V}, \mathcal{E})$}
\Ensure{Una planificación $f$}
\While{$\exists v \in \mathcal{V}$ \emph{no completada}}
	\State $t \gets$ Obtener conjunto de tareas listas, no planificadas, con padres planificados;
	\State \Call{Planificacion}{$tasks$};
\EndWhile
\Procedure{Planificacion}{$availTasks$}
	\While{$\exists t \in availTasks$ no planificadas}
		\For{$t \in availTasks$}
			\State $res \gets$ Obtener recursos disponibles para $t$;
			\For{$r \in res$}
				\State Calcular $ECT(t,r)$;
			\EndFor
			\State $R_T \gets \arg\min_{r \in res}ECT(t,r)$;
		\EndFor
		\State $T \gets \arg\min_{t \in availTasks}ECT(t,R_T)$;
		\State Planificar $T$ en $R_T$;
		\State Remover $T$ de \emph{availTasks};
		\State Actualizar $EAT(R_T)$;
	\EndWhile
\EndProcedure
\end{algorithmic}

\subsection{MaxMin}
\label{alg:max-min}
El algoritmo MaxMin --también propuesto Maheswaran et al. \cite{maheswaran1999dynamic}-- es muy similar al algoritmo MinMin. La diferencia radica en que éste planifica tareas cuyo tiempo mínimo de ejecución es el mayor, de tal modo que se ejecutan primero las tares más \emph{largas}.

El único cambio que se necesita hacer al algoritmo \ref{alg:min-min} es cambiar la línea 14:
\[T \leftarrow \arg\min_{t \in \emph{availTasks}}ECT(t,r);\]
por
\[T \leftarrow \arg\max_{t \in \emph{availTasks}}ECT(t,r);\]
Así, el algoritmo modificado puede verse a continuación: %en el pseudocódigo \ref{alg:max-min}.

\begin{algorithmic}[1]
\Require{Un grafo de flujo de trabajo $w=(\mathcal{V}, \mathcal{E})$}
\Ensure{Una planificación $f$}
\While{$\exists v \in \mathcal{V}$ \emph{no completada}}
	\State $t \gets$ Obtener conjunto de tareas listas, no planificadas, con padres planificados;
	\State \Call{Planificar}{$t$};
\EndWhile
\Procedure{Planificar}{$tareas$}
	\While{$\exists t \in tareas$ no planificadas}
		\For{$t \in tareas$}
			\State $res \gets$ Obtener recursos disponibles para $t$;
			\For{$r \in res$}
				\State Calcular $ECT(t,r)$;
			\EndFor
			\State $R_T \gets \arg\min_{r \in res}ECT(t,r)$;
		\EndFor
		\State $T \gets \arg\max_{t \in availTasks}ECT(t,R_T)$;
		\State Planificar $T$ en $R_T$;
		\State Remover $T$ de \emph{tareas};
		\State Actualizar $EAT(R_T)$;
	\EndWhile
\EndProcedure
\end{algorithmic}

\subsection{Sufragio}
\label{alg:sufferage}
El algoritmo Sufragio\footnote{También conocido como \emph{Sufferage}} \cite{maheswaran1999dynamic} es una variación del algoritmo MinMin, el cual considera el valor \emph{sufragio} para hacer la planificación. Dicho valor es la diferencia entre el menor tiempo de ejecución para una tarea $t$ sobre un conjunto de recursos disponibles y el segundo menor. Se planifica la tarea que tenga el valor del sufragio más alto, por el hecho de que las tareas que son muy sensibles a los cambios de los recursos deben ser planificadas primero. 
%El algoritmo se encuentra en la sección \ref{alg:sufferage} del Apéndice.

\begin{algorithmic}[1]
\Require{Un grafo de flujo de trabajo $w=(\mathcal{V}, \mathcal{E})$}
\Ensure{Una planificación $f$}
\While{$\exists v \in \mathcal{V}$ \emph{no completada}}
	\State $t \gets$ Obtener conjunto de tareas listas, no planificadas, con padres planificados;
	\State \Call{Planificar}{$t$};
\EndWhile
\Procedure{Planificar}{$tareas$}
	\While{$\exists t \in tareas$ no planificadas}
		\For{$t \in tareas$}
			\State $res \gets$ Obtener recursos disponibles para $t$;
			\For{$r \in res$}
				\State Calcular $ECT(t,r)$;
			\EndFor
			\State $R^1_t \gets \arg\min_{r \in res}ECT(t,r)$;
			\State $R^2_t \gets \arg\min_{r \in res, r \ne{R^1_t}}ECT(t,r)$;
			\State $suf_t \gets ECT(t,R^2_t) - ECT(t,R^1_t)$;
		\EndFor
		\State $T \gets \arg\max_{t \in availTasks}suf_t$;
		\State Planificar $T$ en $R^1_T$;
		\State Remover $T$ de \emph{tareas};
		\State Actualizar $EAT(R_T)$;
	\EndWhile
\EndProcedure
\end{algorithmic}

\subsection{HEFT}
\label{alg:heft}
El algoritmo HEFT\footnote{Heterogeneous Earliest-Finish Time} fue propuesto por Topcuoglu et al. \cite{topcuoglu2002performance}. Este algoritmo está dividido en dos fases: (1) priorización de tareas y (2) selección de recursos. En la primera fase, para todas las tareas se calcula un valor llamado $Rango_{Asc}$, que es una estimación del costo de ejecutar una tarea. En la segunda fase, se asigna a cada tarea de la lista un recurso disponible que tenga el tiempo de disponibilidad más corto. Este tiempo se refiere al instante en que un recurso tiene todos los datos necesarios para ejecutar la tarea a planificar.


%Este valor se define de la siguiente forma:

%\begin{equation}
%Rango_{Asc}(t) = \overline{Costo_{comp}(t)} + \max_{n_j \in succ(t)}(\overline{Costo_{comm}(t, n_j)} + Rango_{Asc}(n_j))
%\end{equation}

%Tiempo de ejecución promedio por tarea
El algoritmo HEFT utiliza las fórmulas que vamos a describir a continuación:


Sea $time(T_i, r)$ el tiempo de ejecución de la tarea $T_i$ en el recurso $r$ y sea $R_i$ el conjunto de recursos disponibles para ejecutar la tarea $T_i$. El tiempo de ejecución promedio de la tarea $T_i$ está definido como:
\begin{equation}
\label{ecc:heft1}
\overline{\omega_i} = \frac{\sum_{r \in R_i} time(T_i, r)}{|R_i|}
\end{equation}

Sea $time(e_{ij}, r_i, r_j)$ el tiempo de transferencia de datos entre los recursos $r_i$ y $r_j$ que utilizan el canal $e_{ij}$ para transferir los datos. Sea $R_i$ y $R_j$ los conjuntos de recursos disponibles para ejecutar las tareas $T_i$ y $T_j$, respectivamente. El tiempo promedio de transmisión desde $T_i$ hasta $T_j$ está definido por la siguiente ecuación:
\begin{equation}
\label{ecc:heft2}
\overline{c_{ij}} = \frac{ \sum_{r_i \in R_i, r_j \in R_j} {time(e_{ij}, r_i, r_j)} }{|R_i| |R_j|}
\end{equation}

Cabe recordar que en la primera fase del algoritmo HEFT, las tareas son ordenadas de acuerdo a una función de rango. Esta función está definida por partes. Para una tarea de salida $T$, es decir, una tarea que no tiene sucesores que dependan de ella, el valor de rango es determinado por:
\begin{equation}
\label{ecc:heft3}
Rango(T) = \overline{\omega_i}
\end{equation}

Para las tareas que no son tareas de  salida, el valor de rango se calcula con la siguiente expresión:
\begin{equation}
\label{ecc:heft4}
Rango(T) = \overline{\omega_i} + \max_{T_j \in succ(T_i)} ( \overline{C_{ij}} + Rango(T_j) )
\end{equation}
donde $succ(T_i)$ es el conjunto de los sucesores inmediatos de la tarea $T_i$.

\begin{algorithmic}[1]
\Require{Un grafo de flujo de trabajo $w=(\mathcal{V}, \mathcal{E})$}
\Ensure{Una planificación $f$}
\State Calcular \emph{Tiempo de Ejecución Promedio} \ref{ecc:heft1} para cada tarea $v \in \mathcal{V}$;
\State Calcular \emph{Tiempo de Transferencia de Datos Promedio} \ref{ecc:heft2} entre tareas y sus sucesores;
\State Calcular $Rango_{Asc}$ para cada tarea, de acuerdo a \ref{ecc:heft3} y \ref{ecc:heft4};
\State Ordenar las tareas por $Rango_{Asc}$, en orden decreciente en una lista $Q$;
\While{$Q \ne \varnothing$}
	\State $t \gets$ Remover la primera tarea de $Q$;
	\State $r \gets \arg\min_{r_i \in R}EFT(r_i, t)$ usando \emph{planificación basada en inserciones}
	%\For{$r_i \in R$}
	%	\State Calcular $EFT(r_i, t)$ utilizando la política de \emph{planificación basada en inserciones}
	%\EndFor
	%\State $r \gets$ Encontrar un recurso que pueda ejecutar $t$ en el menor $EFT(r, t)$;
	\State Planificar $t$ en $r$;
\EndWhile
\end{algorithmic}

\subsection{Híbrido}
\label{alg:hybrid}
Este algoritmo combina dos heurísticas: primero agrupa las tareas que tengan dependencias entre sí, de tal modo que haya un balance entre el número de grupos y el número de recursos disponibles para la ejecución de las tareas agrupadas. Luego, con los grupos listos, se utiliza un algoritmo de planificación batch para asignar a cada grupo un recurso. Este algoritmo/heurística fue propuesto por Sakellariou et al \cite{sakellariou2004hybrid}. El algoritmo híbrido utiliza las definiciones de tiempos promedio y función de rango del algoritmo HEFT.

\begin{algorithmic}[1]
\Require{Un grafo de flujo de trabajo $w=(\mathcal{V}, \mathcal{E})$}
\Ensure{Una planificación $f$}
\State Calcular \emph{Peso} de cada tarea y arista, de acuerdo a \ref{ecc:heft1} y \ref{ecc:heft2};
\State Calcular \emph{Rango} para cada tarea, de acuerdo a \ref{ecc:heft3} y \ref{ecc:heft4};
\State Ordenar las tareas por \emph{Rango}, en orden decreciente en una lista $Q$;
\State $i \gets 0$
\State Crear un grupo $G_i$;
\While{$Q \ne \varnothing$}
	\State $t \gets$ Remover la primera tarea de $Q$;
	\If{$t$ tiene una dependencia con una tarea en $G_i$}
		\State $i \gets i + 1$;
		\State Crear un grupo $G_i$;
	\EndIf
	\State Agregar $t$ a $G_i$;
\EndWhile
\State $j \gets 0$;
\While{$j \le i$}
	\State Planificar tareas en $G_i$ usando un algoritmo \emph{batch};
	\State $j \gets j + 1$;
\EndWhile
\end{algorithmic}

\subsection{TANH}
\label{alg:tanh}
Bajaj et al. \cite{bajaj2004improving} proponen un algoritmo basado en dos heurísticas: en la duplicación de tareas y en el agrupamiento de tareas. La primera heurística se basa en que si dos tareas dependen del resultado de una tarea en común y, estas dos tareas se encuentran en recursos diferentes y con un tiempo muerto previo a la ejecución de estas tareas, entonces, se ejecuta de forma redundante la tarea común en los dos recursos diferentes, para que las dos tareas puedan continuar su ejecución de manera independiente y no exista un costo de comunicación de datos entre estos dos recursos. La segunda heurística se refiere a que las tareas con dependencias entre sí se deben ejecutar en los mismos recursos, para que los costos de comunicación de datos sean despreciables.

\begin{algorithmic}[1]
\Require{Un grafo de flujo de trabajo $w=(\mathcal{V}, \mathcal{E})$}
\Ensure{Una planificación $f$}
\State Calcular \emph{Parámetros} para cada nodo tarea;
\State Agrupar tareas del flujo de trabajo;
\If{Número de clusters $\ge$ Número de recursos disponibles}
	\State Reducir el Número de clusters al Número de recursos disponibles;
\Else
	\State Ejecutar duplicación de tareas;
\EndIf
\end{algorithmic}

%\section{Algoritmos de Calidad en el Servicio}
%\subsection{BackTracking}

%\subsection{Deadline Distribution}
%\subsection{CTC}
%\subsection{Deadline-MDP}
%\subsection{LOSS/GAIN}

%\section{Metaheurísticos}
%\subsection{Algoritmo Genético}

\subsection{GRASP}
\label{alg:grasp}
El procedimiento de búsqueda aleatoria voraz adaptativa (GRASP por sus siglas en inglés) es una heurística que se utiliza para resolver problemas de optimización combinatoria. De manera voraz, se construyen planificaciones que reduzcan del tiempo de ejecución total evaluando cada uno de los sucesores de una tarea dada. Así, el algoritmo escoge las tareas de manera aleatoria cuya variación en el tiempo no pase de un rango dado. El algoritmo termina hasta que se cumple un criterio de terminación, como un tiempo límite. Este algoritmo fue propuesto por Blythe et al. \cite{blythe2005task}.

\begin{algorithmic}[1]
\Require{Un grafo de flujo de trabajo $w=(\mathcal{V}, \mathcal{E})$}
\Ensure{Una planificación $f$}
\While{Criterio de terminación no satisfactorio}
	\State $plan \gets$ \Call{crearPlanificacion}{$w$};
	\If{$plan$ es mejor que $mejorPlanificacion$}
		\State $mejorPlanificacion \gets plan$;
	\EndIf
\EndWhile
\Procedure{crearPlanificacion}{$flujo$}
	\State $solucion \gets$ \Call{construirSolucion}{$flujo$};
	\State $nSolucion \gets$ \Call{busquedaLocal}{$solucion$};
	\If{$nSolucion$ es mejor que $solucion$}
		\State \textbf{return} $nSolucion$;
	\EndIf
	\State \textbf{return} $solucion$;
\EndProcedure
\Procedure{construirSolucion}{$flujo$}
	\While{planificación no completada}
		\State $T \gets$ Obtener tareas listas sin asignar;
		\State Crear RCL para cada $t \in T$;
		\State $subSolucion \gets$ Seleccionar recurso \emph{aleatoriamente} para cada $t \in T$ de su RCL;
		\State $solucion \gets solucion \cup subSolucion$;
		\State Actualizar información para futuros RCL;
	\EndWhile
	\State \textbf{return} $solucion$;
\EndProcedure
\Procedure{busquedaLocal}{$solucion$}
	\State $nSolucion \gets$ Encontrar una solución local óptima;
	\State \textbf{return} $nSolucion$;
\EndProcedure
\end{algorithmic}

%\subsection{Recocido simulado}
