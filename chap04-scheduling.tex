\chapter{Planificación}
La planificación
%\footnote{También conocida en la literatura especializada como \emph{scheduling}}
es el proceso de asignación de recursos a tareas, de tal modo que se define un orden de ejecución de las tareas, teniendo lugar diferentes combinaciones de recursos y tareas. En este capítulo se define de la forma más elemental el problema de la planificación para estudiar sus propiedades y también se explica la relación de este problema fundamental con el problema de planificación de flujos de trabajo.


\section{Definición del problema}
\label{secc:scheduling_problem}
De acuerdo a  Ullman et al. \cite{ullman1975np}, el problema de la planificación se define de la siguiente manera: 

%al parecer hay que usar la definición extendida de Ullman para que muestre a la planificación como una función f
\begin{defn}
El \textbf{problema de la planificación} está compuesto por:
\begin{enumerate}
\item Un conjunto de tareas $S = \{ J_1, J_2, \dots, J_n \}$
\item Un ordenamiento parcial $\prec$ sobre $S$
\item Una función de costo $U: S \mapsto \mathbb{Z}^{+}$, la cual indica el tiempo que tarda en completarse cada una de las tareas en $S$
\item Un número de computadoras (procesadores) $k$
\end{enumerate}
\end{defn}

Para Ullman et al. \cite{ullman1975np}, el objetivo del problema de la planificación es \emph{minimizar} el tiempo total de ejecución, denotado por $t_\text{max}$, respetando el orden parcial definido por $\prec$, asignando tareas de $S$ a los $k$ procesadores.
También, es importante notar que se asume que cuando una computadora ejecuta una tarea, ésta es ejecutada completamente, sin errores o interrupciones.

\section{La complejidad de planificar}
Como puede notarse, hay varias maneras de acomodar las $k$ computadoras para que se ejecuten todas las tareas de $S$. De hecho, Ullman et al. han demostrado que este problema pertenece a la categoría NP-completo \cite{ullman1975np}. Esto signifca que no se ha encontrado un algoritmo que pueda resolver el problema en tiempo polinomial. Entonces, la solución ingenua de probar ordenadamente todas las posibles asignaciones de tareas a computadoras resulta computacionalmente muy costoso.

Así, la forma de atacar estos problemas NP-completos es utilizar métodos de aproximación \cite{leiserson2001introduction} que obtengan soluciones subóptimas o utilizar heurísticas que resuelvan este problema asumiendo ciertas restricciones.

\section{Planificación de flujos de trabajo}
Hasta ahora, se ha mencionado el problema básico de planificación con restricciones. Con ello, se pretende plantear el problema de la planificación de flujos de trabajo y apoyarse en la definición del problema básico de planificación para enumerar propiedades sobre este problema.

Para plantear el problema de la planificación de flujos de trabajo, primero se mostrará que, bajo ciertas condiciones, un flujo de trabajo puede ser reducido a un grafo dirigido acíclico haciendo las transformaciones adecuadas. Luego, se utilizará la definición del problema básico de planificación con restricciones y su semejanza para flujos de trabajo. Finalmente, se hará una descripción de la complejidad de planificar flujos de trabajo.
%tres hechos:
% - todo flujo de trabajo puede reducirse a DAG
% - transforma un DAG a un problema clásico de Ullman
% - menciona que el problema de Ullman es NP-completo
\subsection{Reducción de flujos de trabajo a grafos dirigidos acíclicos}
En el trabajo de Mair et al. \cite{mair2007workflow} se propone un formato para una representación intermedia de flujos de trabajo basada en grafos dirigidos acíclicos, con el fin de transformar una especificación detallada de un flujo de trabajo, escrita en un lenguaje basado en XML llamado AGWL, a esta representación intermedia. Como se vio en el capítulo \ref{chap:workflows}, un flujo de trabajo puede ser interpretado desde varias perspectivas. Sin embargo, un grafo dirigido acíclico resume en pocos elementos (nodos y aristas), las tareas y las dependencias que deben ser cumplidas durante la planificación. Es por esta razón que es deseable reducir un flujo de trabajo, interpretado desde cualquier perspectiva, a un grafo dirigido acíclico.

%reflejada en las distintas formas de representación que los sistemas de administración de flujos de trabajo utilizan para operar
Cabe aclarar que, como se vio en el primer capítulo de este trabajo, aún no existe un consenso sobre qué representa un flujo de trabajo \cite{van2003workflow}, debido a las diferentes perspectivas de interpretación de un flujo. En el caso del lenguaje AGWL, éste cuenta con facilidades para poder expresar flujos de trabajo condicionales, a saber: construcciones if, while, for y parallel. Estas facilidades permiten expresar una gran variedad de flujos de trabajo. Sin embargo, para poder estudiar los flujos de trabajo con estas construcciones, se requieren de mecanismos de predicción o estimación de posibles rutas de ejecución del flujo de trabajo. En algunos sistemas de administración de ejecución de flujos de trabajo, como Pegasus \cite{deelman2005pegasus}, aplican desenroscado de bucles como técnica de estimación de la ruta de ejecución del flujo para transformar el flujo de trabajo condicional en un grafo dirigido acíclico. Sin embargo, el estudio de estos mecanismos está fuera del alcance de este trabajo.

%Aunque no existe un consenso general sobre cuál es una definición completa de un flujo de trabajo \cite{van2003workflow}, el lenguaje AGWL es una especificación suficientemente flexible para expresar una gran cantidad de flujos de trabajo, porque con las construcciones if, while, for, parallel, secuence son adecuadas para expresar vastos flujos.

%El hecho importante a recalcar es que los grafos dirigidos acíclicos sirven como representación para utilizarlos en los algoritmos de planificación.

De esta forma, es posible asumir, tanto para el estudio del problema de planificación de flujos de trabajo como para los algoritmos de planificación, que se tendrá como entrada la representación del flujo de trabajo en forma de un grafo dirigido acíclico.

\subsection{Definición del problema de planificación de flujos de trabajo}
Una vez que se ha establecido a los grafos dirigidos acíclicos como nuestra representación básica de flujos de trabajo, se definirá el problema que conlleva asignar recursos a las tareas del flujo.

Con el fin de no perder generalidad, tomaremos las definiciones de Wieczorek-Prodan \cite{wieczorek2008taxonomies} para definir el problema de la planificación de flujos de trabajo:

\begin{defn}
Un \textbf{flujo de trabajo} es un grafo dirigido $w \in \mathcal{W}, w = (\mathcal{V},\mathcal{E})$, compuesto de un conjunto de nodos $\mathcal{V}$ que representan tareas $ \tau \in \mathcal{T}$ y un conjunto de aristas $\mathcal{E}$ que representan transferencias de datos $ \rho \in \mathcal{D}$.
\end{defn}

Hay que notar que en la definición anterior que la forma en que se relacionan las tareas y las transferencias de datos con los nodos y aristas del grafo está determinada por la representación del flujo de trabajo.
%\emph{concreta}
%creo que esto requiere más trabajo.

Ahora, se definirán los recursos de cómputo en donde se ejecuta el flujo. 
%La definición de Wieczorek-Prodan habla de grids, pero es fácilmente aplicable a otros enfoques de cómputo.

\begin{defn}
Un \textbf{servicio} es una entidad de cómputo que puede ejecutar una tarea $\tau \in \mathcal{T}$. El conjunto de todos los servicios disponibles para ejecutar el flujo de trabajo está denotado por $\mathcal{S}$.
\end{defn}

De este modo, la planificación es definida como una función:

\begin{defn}
La \textbf{planificación de un flujo de trabajo} $w$ es una función $ f: \mathcal{T} \mapsto \mathcal{S}$ que asigna servicios a las tareas del flujo. El conjunto que contiene todas las posibles planificaciones de $w$ es denotado por $\mathcal{F}$.
\end{defn}

Cada posible planificación determina un costo de ejecución. A continuación, definiremos el modelo de costo determinado por la utilización de los servicios.

\begin{defn}
Un \textbf{modelo de costos} es un conjunto de criterios $C = \{c_1, c_2, \dots, c_n\}$ que determinan las restricciones en las que se debe ejecutar una planificación, por ejemplo, un límite en el tiempo de ejecución, en el costo monetario o la tolerancia a fallos, entre otros.
\end{defn}

\begin{defn}
Para cada criterio $c_i \in C$, existe una \textbf{función de costo parcial} $\Theta_i : \mathcal{S} \mapsto \mathbb{R}$, en la que a cada servicio disponible para ejecutar el flujo de trabajo se le asocia un costo por ejecutar dicho servicio con las restricciones dictadas con el criterio $c_i$.
\end{defn}

\begin{defn}
Para cada criterio $c_i \in C$, existe una \textbf{función de costo total} $\Delta_i : \mathcal{W} \times \mathcal{F} \mapsto \mathbb{R}$, que asigna un flujo de trabajo $w$ planificado por $f$ un costo basado en los costos parciales determinados por los servicios utlizados para ejecutar las tareas del flujo.
\end{defn}

El objetivo del problema de la planificación de flujos de trabajo es encontrar la planificación $f$ que minimice las funciones de costo total $\Delta_i$, $1 \le i \le n$.

\subsection{Complejidad computacional de la planificación de flujos de trabajo}
%Después de haber enunciado varias definiciones para definir el problema de la planificación de flujos de trabajo
Ahora, se hará una analogía con el problema básico de planificación visto en la sección \ref{secc:scheduling_problem} para estudiar su complejidad computacional. La analogía es la siguiente:


\begin{enumerate}
\item El conjunto de tareas $S$ equivale a las tareas $\mathcal{T}$ descritas por el flujo de trabajo.

\item El ordenamiento parcial $\prec$ está representado tanto por las dependencias de datos $\mathcal{D}$ y las aristas $\mathcal{E}$ que representan el control de flujo que deben respetarse para ejecutar el flujo.

\item Las $k$ computadoras equivalen al conjunto de servicios $\mathcal{S}$ disponibles para ejecutar el flujo.

\item La función de costo $U$ tiene una equivalencia implícita. El problema básico asume que conocemos apriori el tiempo de ejecución de una tarea. Sin embargo, es muy frecuente que sólo tengamos una estimación del tiempo de ejecución. Por otro lado, podemos establecer una función auxiliar que relacione las tareas con los servicios. Dicha relación es la función de planificación $f$. En efecto, ejecutar una tarea en un servicio genera un costo, determinado por las funciones parciales y totales. Por lo pronto, estableceremos una relación proporcional entre costo y tiempo, con el fin de mostrar que existe una equivalencia entre la función de costo $W$ del problema básico y el modelo de costo de un flujo de trabajo.
\end{enumerate}

Con lo anterior, se ha mostrado que el problema de la planificación de flujos de trabajo es equivalente al problema básico de planificación. Entonces, se puede inferir que el problema de la planificación de flujos de trabajo pertenece a la categoría de problemas NP-completos.

%tareas -> servicios -> costo <=> tiempo