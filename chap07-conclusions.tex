\chapter{Conclusiones}

%np-completo: no hay solución polinomial
%no hay un scheduler que pueda cambiar entre algoritmos de calendarización
%es muy difícil hacer un modelo general
%no es una clasificación defininitva
%stocastic based approach no son estudiados, pero debería verlos
%taboo search

En este trabajo hemos revisado el problema de calendarizar flujos de trabajo. Primero se describió el concepto de flujo de trabajo, ilustrándolo con algunos ejemplos de aplicación. También vimos que la descripción de las tareas y las dependencias entre ellas juegan un papel muy importante a la hora de planear y administrar los recursos asigados a cada tarea del flujo de trabajo.

Más adelante, se estableció que los flujos de trabajo son ejecutados en sistemas de cómputo distribuido, los cuales clasificamos en clusters, grids y nubes, de acuerdo a la forma en trabajan para un objetivo común.

Después se estudió la complejidad de calendarizar flujos de trabajo comparando este problema con otro problema de calendarización más general. Como este problema es NP-completo, se han propuesto varias soluciones heurísticas y metaheurísticas que resuelven el problema bajo ciertas circunstancias.

También, con el objetivo de comprender las diferentes heurísticas propuestas para calendarizar flujos de trabajo, se propuso un modelo para describir de la forma más general posible un flujo de trabajo, basados en el trabajo de Wieckzorek et al..

Con el modelo establecido, se realizó una clasificación de los algoritmos de calendarización, la cual está basada en el trabajo de Yu et al.; la aportación de este trabajo fue dar otro punto de visa sobre la forma de clasificar las metaheurísticas, ya que éstas, dependiendo de cómo son programadas, pueden hacer optimizaciones estilo mejor esfuerzo o buscar soluciones que cumplan con restricciones de calidad en el servicio.

Finalmente, se describieron algunos sistemas de administración de flujos de trabajo y los algoritmos de calendarización que utilizan estos sistemas, clasificandolos de acuerdo a la orientación que tienen los algoritmos para calendarizar. Esto es, si la calendarización está pensada para utilizarse en clusters, grids o nubes. Como se puede notar, en estos sistemas puede verse claramente la unión de las dos teorías sobre enfoques de cómputo distribuidos y algoritmos de calendarización.

