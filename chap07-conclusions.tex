\chapter{Conclusiones}

%np-completo: no hay solución polinomial
%no hay un scheduler que pueda cambiar entre algoritmos de calendarización
%es muy difícil hacer un modelo general
%no es una clasificación defininitva
%stocastic based approach no son estudiados, pero debería verlos
%taboo search

En este trabajo hemos revisado el problema de planificar flujos de trabajo. Primero se describió el concepto de flujo de trabajo, ilustrándolo con algunos ejemplos de aplicación. También vimos que la descripción de las tareas y las dependencias entre ellas juegan un papel muy importante a la hora de planear y administrar los recursos asigados a cada tarea del flujo de trabajo.

Más adelante, se estableció que los flujos de trabajo son ejecutados en sistemas de cómputo distribuido, los cuales clasificamos en clusters, grids y nubes, de acuerdo a la forma en trabajan para un objetivo común.

Después se estudió la complejidad de planificar flujos de trabajo comparando este problema con otro problema de planificación más general, orientado a ciencias en computación. Como este problema es NP-completo, se han propuesto varias soluciones heurísticas y metaheurísticas que resuelven el problema bajo ciertas circunstancias.

También, con el objetivo de comprender las diferentes heurísticas propuestas para planificar flujos de trabajo, se utilizó un modelo propuesto en el trabajo elaborado por Wieckzorek et al. para describir de la forma más general posible un flujo de trabajo.

Con el modelo establecido, se realizó una clasificación de los algoritmos de planificación, la cual está basada en el trabajo de Yu et al.; la aportación de este trabajo fue dar otro punto de vista sobre la forma de clasificar las metaheurísticas, ya que éstas, dependiendo de cómo son programadas, pueden hacer optimizaciones estilo mejor esfuerzo o buscar soluciones que cumplan con restricciones de calidad en el servicio.

Finalmente, se describieron algunos sistemas de administración de flujos de trabajo y los algoritmos de planificación que utilizan estos sistemas, clasificandolos de acuerdo a la orientación que tienen los algoritmos para planificar. Esto es, si la planificación está pensada para utilizarse en clusters, grids o nubes. Como se puede notar, en estos sistemas puede verse claramente la unión de las dos teorías sobre enfoques de cómputo distribuidos y algoritmos de planificación.

\section{Retos}
La planificación es una pieza clave para ejecutar flujos de trabajo en entornos distribuido. Sin embargo, existen otras cuestiones que deben ser tomadas en cuenta para utilizar estos algoritmos en un sistema real. La primera de ellas es la tolerancia a fallos. Su importancia radica en que en presencia de fallas se puede replanificar el flujo o tratar de recuperar los resultados o el recurso fallido. Sin duda, este es un tema de investigación muy activo en el área de sistemas distribuidos. Otro aspecto a tomar en cuenta es la seguridad, dado que estos sistemas pueden procesar información sensible y, además, presentan una complejidad exhorbitante, lo cual hace que el monitoreo del sistema de administración de flujos de trabajo sea un tema a tratar.

\section{Trabajo futuro}
Como trabajo futuro, se hará un algoritmo de planificación de flujos de trabajo que pueda manejar restricciones en calidad en el servicio, con especial orientación al enfoque de cómputo distribuido de la nube, ya que, como se vió en este trabajo, hay una gran área de investigación para explorar y diseñar algoritmos que tengan en cuenta las características únicas de las nubes.