\documentclass[letterpaper, 12pt]{report}
\usepackage[utf8]{inputenc}
\usepackage[spanish]{babel}
\usepackage{cite,url,graphicx,amsmath,enumitem}
\usepackage{fullpage}
\usepackage{authblk}

%para definir bibliografia en referencias
\usepackage{etoolbox}
\makeatletter
\patchcmd{\thebibliography}{%
  \chapter*{\bibname}\@mkboth{\MakeUppercase\bibname}{\MakeUppercase\bibname}}{%
  \chapter*{Referencias}}{}{}
\makeatother

\title{Estudio de algoritmos de calendarización para flujos de trabajo}
\author{Fernando Aguilar Reyes}

\affil{Tesina para obtener el título de Ingeniero en Computación \\
       Instituto Tecnológico Autónomo de México \\
     Río Hondo \#1, Progreso Tizapán, Del. Álvaro Obregón, 01080 \\
     México, Distrito Federal \\}


%interlineado
\renewcommand{\baselinestretch}{1.0}

\begin{document}

\maketitle


\chapter{Introducción}

Un flujo de trabajo es un conjunto de pasos que modelan la ejecución de un proceso \cite{gutierrez2012agent}. En la literatura especializada, los flujos de trabajos también son conocidos como \emph{workflows}. En particular, se estudian a los flujos de trabajo utilizados para vislumbrar la ejecución de un proceso de cómputo. A continuación, se muestran algunos ejemplos de flujos de trabajo.



\renewcommand*{\bibname}{\chapter*{Referencias}}
\bibliographystyle{plain}
\bibliography{propuesta}

\end{document}